\section{Related Work}
\label{sec:related}
\par Box-office forecasting is a challenging but import task for movie contributors in their decision making process. Empirical studies of the determinants of box office revenues have mostly focused on post-production factors - that is, ones known after the film has been completed and/or released with audience reactions. Early in \cite{liu2006word} shows that word of mouth (WOM) can help explain box office revenue. The results of experiments from \cite{kim2015box} show that the utilization of SNS data can improve he forecasting accuracies of machine learning-based algorithms and their combination. \cite{hunter2016predicting} regard the textual and content analysis of the screenplays of films as chief among the pre-production factors. \cite{hur2016box} presents new box-office forecasting models to enhance the forecasting accuracy by utilizing review sentiments and employing non-linear machine learning algorithms.

\par Comparing with previous work, \system succeed in integrating unstructured reviews with structured information to enhance and extend application of system. We first extract and qualify sentiment which is core part of audience react information. Specially, we choose more state-of -art techniques (e.g LSTM\cite{katiyar2016investigating}, Heterogenous Network\cite{sun2012when}). It is hard to overemphasize the value of NLP in system. Many study \cite{kim2015box,hur2016box,tang2015target-dependent} focus on single point of film industry taking the IMDB data as the foundation. Instead, we make hard to integrate many aspects which is the distinctive features of the Chinese film market, as well as give interpretable visualized result for follow-up decision. In addition, dynamic analyse of tendency and impact also employ time series model and statistical analysis. To the best of knowledge, \system is a comprehensive solution to overcome tradition difficulty on film data analysis currently.

\section{Conclusion}
\label{sec:conclu}
The return on investment (ROI) analysis about movies is a challenging but important task for movie investors in their decision making process. For investors, they are expected to get high box-office revenue with appropriate investments. Apart from movie special effects, most of investments are used to remuneration for movie actors(actress). Due to complicated factors such as audience reactions, screenplay, film types etc, it's not easy for investors to estimate upfront ROI, which makes film investment a gamble. Although most current research works aim to predict box-office revenue more specific, they can not provide investors more direct profits information about the success of the movie they would like to invest.
\par In this paper, we design and implement an integreated system, called Film Knowledge Analysis Platform (a.k.a FKAP), that aims to do some analysis of movies for investors. FKAP provides various modules for users to predict box-office, capture public opinion about a film, assess the value of the actor, monitor the changes of the ratio of box-office about actors, assess actor replacement and so on.
\section{Acknowledgements}
The work was support by Amoy International Big Data 2017. We thank Xiamen Municipal Bureau of Economy and Information Technology for organization , and Xiamen Huli District Government for sponsorship. Finally, we thank the 2017 AIBD activity for a fun and exciting competition. 